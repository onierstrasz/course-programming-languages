\documentclass[12pt]{amsart}
\usepackage{geometry}                % See geometry.pdf to learn the layout options. There are lots.
\geometry{a4paper}                   % ... or a4paper or a5paper or ... 
%\geometry{landscape}                % Activate for for rotated page geometry
%\usepackage[parfill]{parskip}    % Activate to begin paragraphs with an empty line rather than an indent
\usepackage{graphicx}
\usepackage{amssymb}
\usepackage{epstopdf}
\usepackage{prooftree}

\begin{document}
\thispagestyle{empty}
\pagestyle{empty}

% ----------------------------------------
\section*{Product Introduction}
\[
\begin{prooftree}
n: N, m: M
\justifies
\langle n, m \rangle : N \times M
\end{prooftree}
\]
\newpage
% ----------------------------------------
\section*{Product Elimination}
\[
\begin{prooftree}
e : N \times M
\justifies
\pi_1(e) : N, \pi_2(e) : M
\end{prooftree}
\]
\newpage
% ----------------------------------------
\section*{Function Introduction}
\[
\begin{prooftree}
x : D \vdash e : C
\justifies
\lambda x.e : D \rightarrow C
\end{prooftree}
\]
\newpage
% ----------------------------------------
\section*{Function Elimination}
\[
\begin{prooftree}
f : D \rightarrow C, v : D
\justifies
f(v) : C
\end{prooftree}
\]
\newpage
% ----------------------------------------
\section*{Record Introduction}
\[
\begin{prooftree}
\alpha_i : A, e_i : T_i
\justifies
\{ \alpha_1 \mapsto e_1, \cdots , \alpha_n \mapsto e_n \} : \{ \alpha_1 : T_1, \cdots , \alpha_n : T_n \}
\using \textit{for}\ i=1..n
\end{prooftree}
\]
\newpage
% ----------------------------------------
\section*{Record Elimination}
\[
\begin{prooftree}
e : \{ \alpha_1 : T_1, \cdots , \alpha_n : T_n \}
\justifies
e.\alpha_i : T_i
\using \textit{for}\ i=1..n
\end{prooftree}
\]
\newpage
% ----------------------------------------
\section*{Applying the Rules I}
\[
\textrm{make-point} : \textrm{Integer} \times \textrm{Integer} \rightarrow \textrm{Point}
\]
\newpage
% ----------------------------------------
\section*{Applying the Rules II}
\[
\begin{array}{ll}
\textrm{make-point} = & \lambda(e : \textrm{Integer} \times \textrm{Integer} ) . \{ \\
& x \mapsto \pi_1(e), \\
& y \mapsto \pi_2(e), \\
& \textrm{equal} \mapsto \lambda(p : \textrm{Point}).(\pi_1(e) = p.x \wedge \pi_2(e) = p.y) \}
\end{array}
\]
\newpage
% ----------------------------------------
\section*{Applying the Rules III}
\[
\begin{prooftree}
	\begin{prooftree}
		\begin{prooftree}
		e : \textrm{Integer} \times \textrm{Integer}
		\justifies
		\pi_1(e) : \textrm{Integer},
		\pi_2(e) : \textrm{Integer}
		\end{prooftree}
	\justifies
	...
	\end{prooftree}
	\begin{prooftree}
		\begin{prooftree}
		p : \textrm{Point}
		\justifies
		p.x : \textrm{Integer},
		p.y : \textrm{Integer}
		\end{prooftree}
	\justifies
	...
	\end{prooftree}
\justifies
\textrm{make-point} : \textrm{Integer} \times \textrm{Integer} \rightarrow \textrm{Point}
\end{prooftree}
\]
\newpage
% ----------------------------------------
\section*{Existential Object Encoding}
\newcommand{\Integer}{\textrm{Integer}}
\newcommand{\Point}{\textrm{Point}}
\newcommand{\Boolean}{\textrm{Boolean}}
\newcommand{\rep}{\textrm{rep}}
\[
\begin{array}{lll}
\Point = \exists \rep . (\rep \times & \{ & x : \rep \rightarrow \Integer; \\
&& y : \rep \rightarrow \Integer; \\
&& \textrm{equal} : \rep \times \rep \rightarrow \Boolean\\
&\})
\end{array}
\]
\newpage
% ----------------------------------------
\section*{An Existential Point}
%\newcommand{\rep}{\textrm{rep}}
\[
\begin{array}{lll}
\textrm{aPoint} = \langle\langle 2, 3 \rangle, &\{& x \mapsto \lambda(s : \rep).\pi_1(s),\\
&& y \mapsto \lambda(s : \rep).\pi_2(s),\\
&&\textrm{equal} \mapsto \lambda(p : \rep \times \rep). \\
&& \qquad (\pi_1(\pi_1(p)) = (\pi_1(\pi_2(p)) \\
&& \qquad \wedge \pi_2(\pi_1(p)) = \pi_2(\pi_2(p))) \\
&\}\rangle
\end{array}
\]
\newpage
% ----------------------------------------
\section*{Functional Object Encoding}
%\newcommand{\Integer}{\textrm{Integer}}
%\newcommand{\Point}{\textrm{Point}}
%\newcommand{\Boolean}{\textrm{Boolean}}
\newcommand{\pnt}{\textrm{pnt}}
\[
\begin{array}{lll}
% \Point = \mu ~ \pnt . & \{ & x : \rightarrow \Integer; \\
% && y : \rightarrow \Integer; \\
\Point = \mu ~ \pnt . & \{ & x : \Integer; \\
&& y : \Integer; \\
&& \textrm{equal} : \pnt \rightarrow \Boolean\\
&\})
\end{array}
\]
\newpage
% ----------------------------------------
\section*{A Functional Object}
%\newcommand{\Point}{\textrm{Point}}
\[
\begin{array}{rl}
\textbf{let}~xv = 2, yv = 3~\textbf{in} \\
\textrm{aPoint} =  \{& x \mapsto xv,\\
& y \mapsto yv,\\
&\textrm{equal} \mapsto \lambda(p : \Point). \\
& \qquad (xv = p.x \wedge yv = p.y) \\
\}
\end{array}
\]
\newpage
% ----------------------------------------
\section*{Objects and Recursion}
%\newcommand{\Point}{\textrm{Point}}
\newcommand{\self}{\textrm{self}}
\[
\begin{array}{rl}
\textbf{let}~xv = 2, yv = 3~\textbf{in} \\
\textrm{aPoint} = \mu~\self. \{ & x \mapsto xv,\\
& y \mapsto yv,\\
&\textrm{equal} \mapsto \lambda(p : \Point). \\
& \qquad (\self.x = p.x \wedge \self.y = p.y) \\
\}
\end{array}
\]
\newpage
% ----------------------------------------
\section*{Object Calculus}
%\newcommand{\self}{\textrm{self}}
\[
\begin{array}{rll}
\textrm{aPoint} = [
& x = \varsigma(\self) 2, \\
& y = \varsigma(\self) 3, \\
& \textrm{equal} = \varsigma(\self) \lambda(p) & (~\self.x = p.x \\
&&\wedge~\self.y = p.y) \\
]
\end{array}
\]
\newpage
% ----------------------------------------
\section*{Generators and fixpoints}
%\newcommand{\Point}{\textrm{Point}}
%\newcommand{\self}{\textrm{self}}
\[
\begin{array}{rl}
\textrm{aPoint} = \mu~\self. \{ & x \mapsto 2,\\
& y \mapsto 3,\\
&\textrm{equal} \mapsto \lambda(p : \Point). \\
& \qquad (\self.x = p.x \wedge \self.y = p.y) \\
\}
\end{array}
\]
\newpage
% ----------------------------------------
\section*{Generators and fixpoints II}
%\newcommand{\Point}{\textrm{Point}}
%\newcommand{\self}{\textrm{self}}
\newcommand{\Y}{\textbf{Y}}
\[
\begin{array}{rlrl}
\textrm{genAPoint} &=& \lambda~\self. \{ & x \mapsto 2,\\
&&& y \mapsto 3,\\
&&&\textrm{equal} \mapsto \lambda(p : \Point). \\
&&& \qquad (\self.x = p.x \wedge \self.y = p.y) \\
&&\} \\
\textrm{aPoint} &=& \multicolumn{2}{l}{\Y~\textrm{genAPoint}} \\
&\Rightarrow& \multicolumn{2}{l}{\textrm{genAPoint(genAPoint(genAPoint($\cdots$)))}}\\
\end{array}
\]
\newpage
% ----------------------------------------
\section*{Recursive Types I}
%\newcommand{\Point}{\textrm{Point}}
\newcommand{\GenPoint}{\textrm{GenPoint}}
%\newcommand{\Integer}{\textrm{Integer}}
%\newcommand{\Boolean}{\textrm{Boolean}}
\newcommand{\equal}{\textrm{equal}}
%\newcommand{\self}{\textrm{self}}
%\newcommand{\Y}{\textbf{Y}}
\[
\begin{array}{rcl}
\Point &=& \mu \sigma . \{ x : \Integer; y: \Integer; \equal : \sigma \rightarrow \Boolean \}
\end{array}
\]
\newpage
% ----------------------------------------
\section*{Recursive Types II}
%\newcommand{\Point}{\textrm{Point}}
%\newcommand{\GenPoint}{\textrm{GenPoint}}
%\newcommand{\Integer}{\textrm{Integer}}
%\newcommand{\Boolean}{\textrm{Boolean}}
%\newcommand{\equal}{\textrm{equal}}
%\newcommand{\self}{\textrm{self}}
%\newcommand{\Y}{\textbf{Y}}
\[
\begin{array}{rcl}
\GenPoint &=& \lambda \sigma . \{ x : \Integer; y: \Integer; \equal : \sigma \rightarrow \Boolean \} \\
\Point &=& \Y ~ \GenPoint \\
&\Rightarrow& \GenPoint[\GenPoint[\GenPoint[\cdots]]]
\end{array}
\]
\newpage
% ----------------------------------------
\section*{Function Subtype I}
\[
\begin{array}{l}
f_Y : D_Y \rightarrow C_Y \\ 
f_X : D_X \rightarrow C_X
\end{array}
\]
\newpage
% ----------------------------------------
\section*{Function Subtype II}
\[
\begin{prooftree}
D_X <: D_Y, C_Y <: C_X
\justifies
D_Y \rightarrow C_Y <: D_X \rightarrow C_X
\end{prooftree}
\]
\newpage
% ----------------------------------------
\section*{Record extension}
\[
\begin{prooftree}
a_1, \cdots a_k, \cdots a_n : A
\justifies
\{ a_1 : T_1, \cdots a_n : T_n \} <: \{ a_1 : T_1, \cdots a_ : T_k \}
\using
\textit{for $1 <= k <= n$}
\end{prooftree}
\]
\newpage
% ----------------------------------------
\section*{Record overriding}
\[
\begin{prooftree}
a_i : A, S_i <: T_i
\justifies
\{ a_i : S_i \} <: \{ a_i : T_i \}
\using
\textit{for $i=1..n$}
\end{prooftree}
\]
\newpage
% ----------------------------------------
\section*{Record subtyping}
\[
\begin{prooftree}
a_i : A, S_1 <: T_1, \cdots S_k <: T_k
\justifies
\{ a_1 : S_1, \cdots a_n : S_n \} <: \{ a_1 : T_1, \cdots a_k : T_k \}
\using
\textit{for $1 <= k <= n$}
\end{prooftree}
\]
\newpage
% ----------------------------------------
\section*{Recursive subtype}
\[
\begin{prooftree}
\sigma <: \tau \vdash S <: T
\justifies
\mu \sigma . S <: \mu \tau . T
\using
\textit{$\sigma$ free only in $S$, $\tau$ free only in T}
\end{prooftree}
\]
\newpage
% ----------------------------------------
\end{document}

