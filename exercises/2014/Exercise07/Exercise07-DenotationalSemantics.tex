\documentclass [11pt, a4wide, twoside]{article}

\usepackage{times}
\usepackage{epsfig}
\usepackage{ifthen}
\usepackage{xspace}
\usepackage{fancyhdr}

\usepackage{sverb}

% solution switch
\newboolean{showsolution}
\setboolean{showsolution}{true}


%layout
\topmargin      -5.0mm
\oddsidemargin  6.0mm
\evensidemargin -6.0mm
\textheight 215.5mm
\textwidth      160.0mm
\parindent        1.0em
\headsep          10.3mm
\headheight        12pt
\lineskip    1pt
\normallineskip     1pt

%header
\lhead{Programming Languages \\ 2021}

\rhead{Prof. O. Nierstrasz\\
Mohammadreza Hazhirpasand, Joel Niklaus}
\lfoot{page \thepage}
\rfoot{\today}
\cfoot{}

\renewcommand{\headrulewidth}{0.1pt}
\renewcommand{\footrulewidth}{0.1pt}

\renewcommand{\thesubsection}{\arabic{subsection}}

%enumeration
\newenvironment{myitemize}{%
     \begin{itemize}
     \setlength{\itemsep}{0cm}}
     {\end{itemize}}

\newenvironment{myenumerate}{%
     \begin{enumerate} \setlength{\itemsep}{0cm}}
     {\end{enumerate}}


%solution
\ifthenelse{\boolean{showsolution}}
   {  \newcommand{\solution}[1]{
   	\noindent\underline{\textbf{Answer:}}\\[2mm]
   	 \textsl{#1}
	 \vspace{10pt}
	 \normalsize
	}
  }
  {  \newcommand{\solution}[1]{} }

\newcounter{exnum}
\def\xexercise{\fontsize{12}{10}\fontseries{bx}\selectfont}
\def\xnormal{\fontseries{m}\fontshape{n}\selectfont}


\newcommand{\exercise}[1]{%
     {\addtocounter{exnum}{1}\vskip 0.8cm{\xexercise \noindent Exercise
\arabic{exnum} (#1)} \xnormal} \vskip 0.3cm} 
 \newcommand{\aufgabe}[1]{
     {\addtocounter{exnum}{1}\vskip 0.8cm{\xexercise \noindent Aufgabe
\arabic{exnum} (#1)} \xnormal} \vskip 0.3cm} 

\pagestyle{fancy}


% ===============ABBREVIATIONS==============================
\newcommand{\eg}{\emph{e.g.,}\xspace}
\newcommand{\ie}{\emph{i.e.,}\xspace}
\newcommand{\etc}{\emph{etc.}\xspace}


\begin{document}

% title
\section*{\ifthenelse{\boolean{showsolution}}{Solution }{}\xspace{}Serie 7 - Denotational Semantics}

% - - - - - - - - - - - - - - - - - - - - - - - - - - - - - - - - - - - - - - - - - - - - - - - - - - - - - - - - - - - - - - - - - - -
\subsection*{Exercise 1}
Answer the following questions:

\begin{myitemize}
\item What is the difference between syntax and semantics?
\solution{Syntax: the arrangement of words and phrases to create well-formed sentences in a language \\
Semantics: the meaning of a word, phrase, sentence or text\\
You can create well-formed sentences (according to the syntax) that don't have a meaning (according to semantics).}

\item How can you specify semantics as mappings from syntax to behaviour?
\solution{This is done by the semantics (operational, denotational, axiomatic or structured operational). They specify an output for each syntactically correct input and how to interpret it ([[input]]). This mapping is of course not a bijection by default.}

\item Does the calculator semantics specify strict or lazy evaluation?
\solution{It is lazy. Operators are always evaluated first, then their arguments.
If you consider the Expression \texttt{E1+E2} and evaluate it, you'll get \texttt{E1 '+' E2}, that is, first '+' is mapped. If it were strict, you would have had \texttt{[[E1+E2]] -> [[ [[E1]] + E2 ]]} etc.}

\item Does the implementation of the calculator semantics use strict or lazy evaluation?
\solution{It is lazy. Same argument as above. If it were strict, the expression \texttt{ee (Plus e1 e2) n} evaluated would be \texttt{ee (Plus (ee e1 n) e2) n}.}

\item Why are semantic functions typically higher oder?
\solution{It's easier to modell environments with high order functions.}
\end{myitemize}

% - - - - - - - - - - - - - - - - - - - - - - - - - - - - - - - - - - - - - - - - - - - - - - - - - - - - - - - - - - - - - - - - - - -
\subsection*{Exercise 2}

Consider a language of decimal numerals. The numeral \textbf{'123'} is intended to denote the number \textbf{123}, and \textbf{'876'} to denote the number \textbf{876}. We define the \emph{syntax} of decimal numerals as:

\begin{small}
\begin{verbatim}
        Numeral ::= Digit
                 | Numeral Digit
        Digit := 0 | 1 | 2 | 3 | 4 | 5 | 6 | 7 | 8 | 9
\end{verbatim}
\end{small}

\noindent Define the semantic functions and the domain of this language. As a test evaluate \textbf{'876'}.
\\\\
\solution{$
value: Numeral \rightarrow Natural\\
value[[Numeral~Digit]]= 10 \times value[[Numeral]] + value[[Digit]]\\
value[[0]]=0, value[[1]]=1, ...\\
$
Test:\\
$ value[[876]]=10 \times value[[87]] + value[[6]] = 10 \times (10 \times value[[8]] + value[[7]]) + value[[6]] = 10 \times (10 \times 8 + 7) + 6 = 876 $
}

% - - - - - - - - - - - - - - - - - - - - - - - - - - - - - - - - - - - - - - - - - - - - - - - - - - - - - - - - - - - - - - - - - - -


\end{document}
