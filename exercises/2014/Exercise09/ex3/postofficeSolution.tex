\begin{verbatim}
<html>
<head>
<title>Post Office</title>
<script type="text/javascript">
// Sample solution; Oscar Nierstrasz 2012.05.02
var customer = {
	name : '(unknown)',
	priority : function() {
		return false;
	},
	// Hint: define toString() to make an object printable 
	toString : function() {
		return this.name;
	}
}

var bob = Object.create(customer);
bob.name = 'Bob';
var carol = Object.create(customer);
carol.name = 'Carol';
var ted = Object.create(customer);
ted.name = 'Ted';
var alice = Object.create(customer);
alice.name = 'Alice';

var special = Object.create(customer);
special.priority = function() {
	return true;
}
var ray = Object.create(special);
ray.name = 'Ray';
var ludwig = Object.create(special);
ludwig.name = 'Ludwig';

// Hint: use JavaScript arrays to represent queues.
var customers = [ bob, carol, ray, ted, ludwig, alice ];

// Hint: make Post Office a closure
var postoffice = (function() {
	var queue = [];
	var counter = [ [], [] ];
	var enterCustomer = function() {
		if (customers.length > 0) {
			var newCustomer = customers.shift();
			if (newCustomer.priority()) {
				queue.unshift(newCustomer);
			} else {
				queue.push(newCustomer);
			}
		}
		updateQueue();
	}
	var updateQueue = function() {
		updateCounter(1);
		updateCounter(2);
		updateDisplay();
	}
	var updateCounter = function(n) {
		if (queue.length > 0) {
			if (counter[n-1].length == 0) {
				counter[n-1].push(queue.shift());
			}
		}
	}
	var updateDisplay = function() {
		// Hint: use Array.join() to concatenate array elements
		show('queue', queue.join(', '));
		show('counter1', counter[0].join(', '));
		show('counter2', counter[1].join(', '));
	}
	var serve = function(n) {
		if (counter[n-1].length > 0) {
			customers.push(counter[n-1].shift());
		}
		updateQueue();
	}
	return {
		enterCustomer : enterCustomer,
		serve : serve
	}
})();

function show(id, text) {
	document.getElementById(id).innerHTML = text;
}
</script>
<style>
table, th, td {
	border: 1px solid black;
}
</style>
</head>
<body>
<h1>Post Office</h1>
<p>Bob, Carol, Ted and Alice are regular customers. Ray and Ludwig are special customers with priority service, and always go to the head of the queue.</p>
<table>
	<tr>
		<td><strong>Queue</strong></td>
		<td id="queue"></td>
	</tr>
	<tr>
		<td><strong>Counter 1</strong></td>
		<td id="counter1"></td>
	</tr>
	<tr>
		<td><strong>Counter 2</strong></td>
		<td id="counter2"></td>
	</tr>
</table>
<button type="button" onclick="postoffice.enterCustomer()">Customer enters</button>
<button type="button" onclick="postoffice.serve(1)">Serve counter 1</button>
<button type="button" onclick="postoffice.serve(2)">Serve counter 2</button>

</body>
</html> 
\end{verbatim}
