\documentclass [11pt, a4wide, twoside]{article}

\usepackage{times}
%\usepackage{epsfig}
\usepackage{ifthen}
\usepackage{xspace}
\usepackage{fancyhdr}
\usepackage{graphicx}
\usepackage[colorlinks,urlcolor=blue]{hyperref}
\newcommand{\yellowbox}[1]{\fcolorbox{yellow}{yellow}{\bfseries\sffamily\scriptsize#1}}
\newcommand{\nb}[2]{{\yellowbox{#1}\yellowbox{#2}}}

\newcommand\todo[1]{\nb{TO DO}{#1}}

% solution switch
\newboolean{showsolution}
\setboolean{showsolution}{true} 
%\setboolean{showsolution}{false}


%layout
\topmargin      -5.0mm
\oddsidemargin  6.0mm
\evensidemargin -6.0mm
\textheight 215.5mm
\textwidth      160.0mm
\parindent        1.0em
\headsep          10.3mm
\headheight        12pt
\lineskip    1pt
\normallineskip     1pt

%header
\lhead{Programming Languages \\ 2021}

\rhead{Prof. O. Nierstrasz\\
Mohammadreza Hazhirpasand, Joel Niklaus}
\lfoot{page \thepage}
\rfoot{\today}
\cfoot{}

\renewcommand{\headrulewidth}{0.1pt}
\renewcommand{\footrulewidth}{0.1pt}

\renewcommand{\thesubsection}{\arabic{subsection}}

%enumeration
\newenvironment{myitemize}{%
     \begin{itemize}
     \setlength{\itemsep}{0cm}}
     {\end{itemize}}

\newenvironment{myenumerate}{%
     \begin{enumerate} \setlength{\itemsep}{0cm}}
     {\end{enumerate}}


%solution
\ifthenelse{\boolean{showsolution}}
   {  \newcommand{\solution}[1]{
   	\noindent\underline{\textbf{Answer:}}\\[2mm]
   	 \textsl{#1}
	 \vspace{10pt}
	 \normalsize
	}
  }
  {  \newcommand{\solution}[1]{} }

\newcounter{exnum}
\def\xexercise{\fontsize{12}{10}\fontseries{bx}\selectfont}
\def\xnormal{\fontseries{m}\fontshape{n}\selectfont}


\newcommand{\exercise}[1]{%
     {\addtocounter{exnum}{1}\vskip 0.8cm{\xexercise \noindent Exercise
\arabic{exnum} (#1)} \xnormal} \vskip 0.3cm} 
 \newcommand{\aufgabe}[1]{
     {\addtocounter{exnum}{1}\vskip 0.8cm{\xexercise \noindent Aufgabe
\arabic{exnum} (#1)} \xnormal} \vskip 0.3cm} 

\pagestyle{fancy}


% ===============ABBREVIATIONS==============================
\newcommand{\eg}{\emph{e.g.,}\xspace}
\newcommand{\ie}{\emph{i.e.,}\xspace}
\newcommand{\etc}{\emph{etc.}\xspace}


\begin{document}

% title
\section*{\ifthenelse{\boolean{showsolution}}{Solution }{}\xspace{}Exam Programming Languages}

% - - - - - - - - - - - - - - - - - - - - - - - - - - - - - - - - - - - - - - - - - - - - - - - - - - - - - - - - - - - - - - - - - - -

\exercise{PostScript}
\noindent
%

%A very junior programmer wanted to write the ``hello world'' program in Postscript. Here is the result: 

\begin{small}
\begin{verbatim}
/box {          
    newpath
    moveto 
    0 150 rlineto
    150 0 rlineto
    0 -150 rlineto
    closepath
    setgray
    fill
} def

0 100 100 box
0.4 200 200 box
0.6 300 300 box
0 setgray

showpage

\end{verbatim}
\end{small}

\newpage
% - - - - - - - - - - - - - - - - - - - - - - - - - - - - - - - - - - - - - - - - - - - - - - - - - - - - - - - - - - - - - - - - - - -

\exercise{Haskell}
\noindent
Write a function in Haskell that checks if a number is prime. A natural number (i.e. 1, 2, 3, 4, 5, 6, etc.) is called a prime number (or a prime) if it has exactly two positive divisors, 1 and the number itself.


\newpage
% - - - - - - - - - - - - - - - - - - - - - - - - - - - - - - - - - - - - - - - - - - - - - - - - - - - - - - - - - - - - - - - - - - -

\exercise{Lambda Calculus}
\noindent
%
Consider the following $\lambda$-expression. Indicate which occurrences of variables are bound and which ones are free in the expressions and reduce it to its normal form if possible.
\\


\texttt{(($\lambda$ a b .~a b c) ($\lambda$ x y z .~z y x))($\lambda$ f .~q f) } \vspace{1cm}

\newpage
% - - - - - - - - - - - - - - - - - - - - - - - - - - - - - - - - - - - - - - - - - - - - - - - - - - - - - - - - - - - - - - - - - - -
\exercise{12 Points}
\noindent What is the output of the following program:

\begin{verbatim}
var alien = {
  kind: 'alien',
  voice: 'ugabuga'
}

var person = {
  kind: 'person',
}

var zack = {};

zack.__proto__ = alien;
console.log(zack.kind);

zack.__proto__ = person;
console.log(zack.kind);

person.__proto__ = alien;
console.log(zack.kind);
console.log(zack.voice);
\end{verbatim}


\newpage
% - - - - - - - - - - - - - - - - - - - - - - - - - - - - - - - - - - - - - - - - - - - - - - - - - - - - - - - - - - - - - - - - - - -

\exercise{Prolog}
Consider the following knowledge base:
\begin{verbatim}
  ...
  loves(vincent,mia). 
  loves(mia, vincent). 
  loves(joe,mia).  
  loves(joe, joe).
  ...
\end{verbatim}

Define the following rules:
\begin{enumerate}
\item potentialCouple(X,Y) - if both X and Y love each other
\item thirdWheel(X,Y,Z) - if both X and Y love each other and Z is in love with one of them
\end{enumerate}

% - - - - - - - - - - - - - - - - - - - - - - - - - - - - - - - - - - - - - - - - - - - - - - - - - - - - - - - - - - - - - - - - - - -
\end{document}