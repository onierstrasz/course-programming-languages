\documentclass [11pt, a4wide, twoside]{article}

\usepackage{times}
\usepackage{epsfig}
\usepackage{ifthen}
\usepackage{xspace}
\usepackage{fancyhdr}
\usepackage{hyperref}

% solution switch
\newboolean{showsolution}
\setboolean{showsolution}{false}


%layout
\topmargin      -5.0mm
\oddsidemargin  6.0mm
\evensidemargin -6.0mm
\textheight 215.5mm
\textwidth      160.0mm
\parindent        1.0em
\headsep          10.3mm
\headheight        12pt
\lineskip    1pt
\normallineskip     1pt

%header
\lhead{Programming Languages \\ 2021}

\rhead{Prof. O. Nierstrasz\\
Mohammadreza Hazhirpasand, Joel Niklaus}
\lfoot{page \thepage}
\rfoot{\today}
\cfoot{}

\renewcommand{\headrulewidth}{0.1pt}
\renewcommand{\footrulewidth}{0.1pt}

\renewcommand{\thesubsection}{\arabic{subsection}}

%enumeration
\newenvironment{myitemize}{%
     \begin{itemize}
     \setlength{\itemsep}{0cm}}
     {\end{itemize}}

\newenvironment{myenumerate}{%
     \begin{enumerate} \setlength{\itemsep}{0cm}}
     {\end{enumerate}}


%solution
\ifthenelse{\boolean{showsolution}}
   {  \newcommand{\solution}[1]{
   	\noindent\underline{\textbf{Answer:}}\\[2mm]
   	 \textsl{#1}
	 \vspace{10pt}
	 \normalsize
	}
  }
  {  \newcommand{\solution}[1]{} }

\newcounter{exnum}
\def\xexercise{\fontsize{12}{10}\fontseries{bx}\selectfont}
\def\xnormal{\fontseries{m}\fontshape{n}\selectfont}


\newcommand{\exercise}[1]{%
     {\addtocounter{exnum}{1}\vskip 0.8cm{\xexercise \noindent Exercise
\arabic{exnum} (#1)} \xnormal} \vskip 0.3cm} 
 \newcommand{\aufgabe}[1]{
     {\addtocounter{exnum}{1}\vskip 0.8cm{\xexercise \noindent Aufgabe
\arabic{exnum} (#1)} \xnormal} \vskip 0.3cm} 

\pagestyle{fancy}


% ===============ABBREVIATIONS==============================
\newcommand{\eg}{\emph{e.g.,}\xspace}
\newcommand{\ie}{\emph{i.e.,}\xspace}
\newcommand{\etc}{\emph{etc.}\xspace}


\begin{document}

% title
\section*{\ifthenelse{\boolean{showsolution}}{Solution\space{}}{}Serie 3 - Haskell}

\textbf{NOTE} Please use the provided \emph{template.hs} file! it contains unit tests for the required functions. Feel free to add additional tests.
% - - - - - - - - - - - - - - - - - - - - - - - - - - - - - - - - - - - - - - - - - - - - - - - - - - - - - - - - - - - - - - - - - - -
\subsection*{Exercise 1}

Write a Haskell function \emph{isPrime} which, given an integer, returns whether or 
not the integer is a prime number.

% - - - - - - - - - - - - - - - - - - - - - - - - - - - - - - - - - - - - - - - - - - - - - - - - - - - - - - - - - - - - - - - - - - -

\subsection*{Exercise 2\footnote{This exercise is an adaption of problem 6 from \url{http://projecteuler.net/}. You can find a ton of cool problems there.}}

The sum of the squares of the first ten natural numbers is,

$1^2 + 2^2 + ... + 10^2 = 385$

The square of the sum of the first ten natural numbers is,

$(1 + 2 + ... + 10)^2 = 55^2 = 3025$

Hence the difference between the sum of the squares of the first ten natural numbers and the square of the sum is $3025 - 385 = 2640.$

Write a Haskell function \emph{diffSquareOfSumAndSumOfSquares} which, given an integer \emph{x}, finds the difference between the square of the sum and the sum of the squares for the first \emph{x} natural numbers.




\solution{and $\equiv$ $\lambda$~x y .~x y x \\
	\indent\indent // if the first argument is \texttt{False}, the function should return \texttt{False}, \ie the first argument; \\
	\indent\indent // if the first argument is \texttt{True}, the function should return the second argument\\
or $\equiv$ $\lambda$~x y .~x x y \\
	\indent\indent // if the first argument is \texttt{True}, the function should return \texttt{True}, \ie the first argument; \\
	\indent\indent // if the first argument is \texttt{False}, the function should return the second argument\\

\texttt{True and False = False}\\
\indent\indent ($\lambda$~x y.~x y x)($\lambda$~x y.~x)($\lambda$~x y.~y) = \\
\indent\indent ($\lambda$~x y.~x)($\lambda$~x y.~y)($\lambda$~x y.~x) = \\
\indent\indent ($\lambda$~x y.~y) $\equiv$ \texttt{False} \\

\texttt{True or False = True}\\
\indent\indent ($\lambda$~x y.~x x y)($\lambda$~x y.~x)($\lambda$~x y.~y) = \\
\indent\indent ($\lambda$~x y.~x)($\lambda$~x y.~x)($\lambda$~x y.~y) = \\
\indent\indent ($\lambda$~x y.~x) $\equiv$ \texttt{True} \\}
% - - - - - - - - - - - - - - - - - - - - - - - - - - - - - - - - - - - - - - - - - - - - - - - - - - - - - - - - - - - - - - - - - - -
\subsection*{Exercise 3}

In this exercise, you are going to implement a set of functions which operate on lists. Their semantics are given below.

\begin{enumerate}
\renewcommand{\theenumi}{\alph{enumi}}
\item Write a function \verb+insertNode+ which adds a new node to the list. The new node should be inserted before the first node with a higher value (we assume that all lists to contain numbers).
\item Write a function \verb+deleteNodes+ which deletes all nodes which satisfy a certain predicate \verb+p+.
\item Write a function \verb+removeDuplicates+ which removes duplicates to get a list with nodes having unique values.
\item Write a function \verb+sumNodes+ which calculates the sum of all nodes of the list.
\item Write a mapping function \verb+mapList+ which applies to each node of the list a given function \texttt{f}, e.g., the \verb+square+ function, and returns a list with the resulting values.
\item Write a function \verb+mergeLists+ which merges two sorted lists to produce one list which is also sorted.\label{lb}
\item Use the function from \ref{lb} to implement a sorting function \verb+sortList+ which sorts a list in ascending order. A \emph{Mergesort} would be adequate in this case.
\end{enumerate}

% - - - - - - - - - - - - - - - - - - - - - - - - - - - - - - - - - - - - - - - - - - - - - - - - - - - - - - - - - - - - - - - - - - -

%\subsection*{Exercise 5}
%
%Define the following functions:
%
%\begin{myitemize}
%	\item Define a function \texttt{nondecreasing} that takes a list \texttt{xs} as parameter and returns \texttt{True} iff the elements of \texttt{xs} are in non-decreasing order.
%
%	\item Define a more general function \texttt{goodlist} that takes a list \texttt{xs} and a predicate \texttt{p} as parameters and returns \texttt{True} iff each adjacent pair of elements in \texttt{xs} satisfies \texttt{p}.
%	
%	\item Now use the function \texttt{goodlist} to define further functions:
%
%		\begin{myitemize}
%			\item A function \texttt{alternating} that takes a list of numbers \texttt{xs} as parameter and returns \texttt{True} iff the elements of \texttt{xs} alternate odd and even.
%
%			\item A function \texttt{partial} that takes a list of numbers \texttt{xs} as parameter and returns \texttt{True} iff each number \texttt{x} in \texttt{xs} is either a multiple or a factor of the number preceding \texttt{x}.
%		\end{myitemize}
%\end{myitemize}
%
%\solution{\vspace{-20pt}\noindent \input{exercise5.tex}}

\end{document}
