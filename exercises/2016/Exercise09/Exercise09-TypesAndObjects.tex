\documentclass [11pt, a4wide, twoside]{article}

\usepackage{times}
\usepackage{epsfig}
\usepackage{ifthen}
\usepackage{xspace}
\usepackage{fancyhdr}
\usepackage{moreverb}
\usepackage{amsmath}

% solution switch
\newboolean{showsolution}
\setboolean{showsolution}{true}
\setboolean{showsolution}{false}



%layout
\topmargin      -5.0mm
\oddsidemargin  6.0mm
\evensidemargin -6.0mm
\textheight 215.5mm
\textwidth      160.0mm
\parindent        1.0em
\headsep          10.3mm
\headheight        12pt
\lineskip    1pt
\normallineskip     1pt

%header
\lhead{Programming Languages \\ 2021}

\rhead{Prof. O. Nierstrasz\\
Mohammadreza Hazhirpasand, Joel Niklaus}
\lfoot{page \thepage}
\rfoot{\today}
\cfoot{}

\renewcommand{\headrulewidth}{0.1pt}
\renewcommand{\footrulewidth}{0.1pt}

\renewcommand{\thesubsection}{\arabic{subsection}}

%enumeration
\newenvironment{myitemize}{%
     \begin{itemize}
     \setlength{\itemsep}{0cm}}
     {\end{itemize}}

\newenvironment{myenumerate}{%
     \begin{enumerate} \setlength{\itemsep}{0cm}}
     {\end{enumerate}}


%solution
\ifthenelse{\boolean{showsolution}}
   {  \newcommand{\solution}[1]{
   	\noindent\underline{\textbf{Answer:}}\\[2mm]
   	 \textsl{#1}
	 \vspace{10pt}
	 \normalsize
	}
  }
  {  \newcommand{\solution}[1]{} }

\newcounter{exnum}
\def\xexercise{\fontsize{12}{10}\fontseries{bx}\selectfont}
\def\xnormal{\fontseries{m}\fontshape{n}\selectfont}


\newcommand{\exercise}[1]{%
     {\addtocounter{exnum}{1}\vskip 0.8cm{\xexercise \noindent Exercise
\arabic{exnum} (#1)} \xnormal} \vskip 0.3cm} 
 \newcommand{\aufgabe}[1]{
     {\addtocounter{exnum}{1}\vskip 0.8cm{\xexercise \noindent Aufgabe
\arabic{exnum} (#1)} \xnormal} \vskip 0.3cm} 

\pagestyle{fancy}


% ===============ABBREVIATIONS==============================
\newcommand{\eg}{\emph{e.g.,}\xspace}
\newcommand{\ie}{\emph{i.e.,}\xspace}
\newcommand{\etc}{\emph{etc.}\xspace}


\begin{document}

% title
\section*{
    \ifthenelse{\boolean{showsolution}}{Solution }{}
    \xspace{}Serie 9 - Objects and Types}

%==============================================================================
\subsection{Theoretical Questions}

\begin{myenumerate}

% \item What is a closure? How is an object like a closure?

% \solution{}

\item How can covariance lead to type errors?

\solution{Because I can get a value in my parameters that do not result in an expected type}

\item How does contravariance support subtyping?

\solution{Because the contravariance parameter type will always end in the correct result subclass since $f_y$ is a superset of $f_x$}


\item What is the difference between subtyping and specialization?

\solution{}


\item Why overloading is practical? Why it is evil?

\solution{}



\item What problems do recursive types pose for subtyping?

\solution{}


\item Why is subtyping a ``universal'' form of polymorphism?

\solution{}


\item How is a class like a family of types? 

\solution{}

\end{myenumerate}

%\subsection{Type generators in Haskell}
%
%Suppose we have the data types Cat and Dog defined as shown below.
%\begin{verbatim}
%data Cat = Cat Int
%    deriving Show
%
%data Dog = Dog Int
%    deriving Show
%\end{verbatim}
%
%\begin{myenumerate}
%\item Explain how type generators are used in Haskell.
%\item Define a type generator \verb$Animal$ that provides the function
%      \verb$mate$.
%\item Generate the animal types for cats and dogs. Implement mate for cats
%      so that it returns a new cat with a value that is the sum of
%      both values of its parents. Implement the mate for dogs so that
%      it returns a new dog with a value that is the product of the
%      values of its parents.  
%\end{myenumerate}

\subsection{Types in Java}
Integers and floats in Java are primitive data types. Since they do not have a
common superclass it is impossible to pass a float where an integer was
expected and vice versa. Implement a new class hierarchy in Java for numbers
that supports the \verb$times$ operation so that the tests in Ex09-2.zip get green.
You are not allowed to use \verb$instanceof$.
% \solution{\input{ex2Sol.tex}}
%\begin{verbatim}
%Float a   = new Float(1.4);
%Integer b = new Integer(10);
%Number c  = new Float(5.6);
%Number d  = new Integer(55);
%
%a.times(d);
%b.times(c);
%
%d.times(a);
%c.times(b);
%
%c.times(d);
%d.times(c);
%\end{verbatim}

\end{document}

