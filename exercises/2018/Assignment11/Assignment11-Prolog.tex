\documentclass [11pt, a4wide, twoside]{article}

\usepackage{times}
\usepackage{epsfig}
\usepackage{ifthen}
\usepackage{xspace}
\usepackage{fancyhdr}
\usepackage{moreverb}

% solution switch
\newboolean{showsolution}
\setboolean{showsolution}{true}


%layout
\topmargin      -5.0mm
\oddsidemargin  6.0mm
\evensidemargin -6.0mm
\textheight 215.5mm
\textwidth      160.0mm
\parindent        1.0em
\headsep          10.3mm
\headheight        12pt
\lineskip    1pt
\normallineskip     1pt

%header
\lhead{Programming Languages \\ 2021}

\rhead{Prof. O. Nierstrasz\\
Mohammadreza Hazhirpasand, Joel Niklaus}
\lfoot{page \thepage}
\rfoot{\today}
\cfoot{}

\renewcommand{\headrulewidth}{0.1pt}
\renewcommand{\footrulewidth}{0.1pt}

\renewcommand{\thesubsection}{\arabic{subsection}}

%enumeration
\newenvironment{myitemize}{%
     \begin{itemize}
     \setlength{\itemsep}{0cm}}
     {\end{itemize}}

\newenvironment{myenumerate}{%
     \begin{enumerate} \setlength{\itemsep}{0cm}}
     {\end{enumerate}}


%solution
\ifthenelse{\boolean{showsolution}}
   {  \newcommand{\solution}[1]{
   	\noindent\underline{\textbf{Answer:}}\\[2mm]
   	 \textsl{#1}
	 \vspace{10pt}
	 \normalsize
	}
  }
  {  \newcommand{\solution}[1]{} }

\newcounter{exnum}
\def\xexercise{\fontsize{12}{10}\fontseries{bx}\selectfont}
\def\xnormal{\fontseries{m}\fontshape{n}\selectfont}


\newcommand{\exercise}[1]{%
     {\addtocounter{exnum}{1}\vskip 0.8cm{\xexercise \noindent Exercise
\arabic{exnum} (#1)} \xnormal} \vskip 0.3cm} 
 \newcommand{\aufgabe}[1]{
     {\addtocounter{exnum}{1}\vskip 0.8cm{\xexercise \noindent Aufgabe
\arabic{exnum} (#1)} \xnormal} \vskip 0.3cm} 

\pagestyle{fancy}


% ===============ABBREVIATIONS==============================
\newcommand{\eg}{\emph{e.g.,}\xspace}
\newcommand{\ie}{\emph{i.e.,}\xspace}
\newcommand{\etc}{\emph{etc.}\xspace}


\begin{document}

% title
\section*{\ifthenelse{\boolean{showsolution}}{Solution }{}\xspace{}Applications of Logic Programming}

% - - - - - - - - - - - - - - - - - - - - - - - - - - - - - - - - - - - - - - - - - - - - - - - - - - - - - - - - - - - - - - - - - - -
\subsection*{Exercise 1}
% - - - - - - - - - - - - - - - - - - - - - - - - - - - - - - - - - - - - - - - - - - - - - - - - - - - - - - - - - - - - - - - - - - -
Create a finite collection of definite clause grammar rules to check whether a sentence is grammatically correct.
%
A sentence can be composed of the following words:
\begin{description}
	\item [\texttt{article}] \texttt{a}, \texttt{the}
	\item [\texttt{noun}] \texttt{girl}, \texttt{boy}
	\item [\texttt{pronoun}] \texttt{that}, \texttt{this}
	\item [\texttt{auxiliary}] \texttt{is}
	\item [\texttt{verb}] \texttt{sleeps}, \texttt{likes}.
\end{description}
%
A sentence must be in one of the following forms \texttt{subject-predicate} and  \texttt{subject-predicate-object}.
%
\begin{itemize}
\item \texttt{subject} is formed out of either an \texttt{article} and a \texttt{noun}, or a \texttt{pronoun}. For example, \texttt{a girl} or \texttt{that}.
\item \texttt{predicate} is either an \texttt{auxiliary} or a \texttt{verb}
\item \texttt{object} is formed out of an \texttt{article} and a \texttt{noun}
\end{itemize}
%
In the \texttt{subject-predicate} form of a sentence, the \texttt{predicate} can be only \texttt{sleeps}.
%
In the \texttt{subject-predicate-object} form of a sentence the verb can be either \texttt{likes} or \texttt{is}. If the \texttt{predicate} is \texttt{likes}, the 		\texttt{subject} is composed of an \texttt{article} and a \texttt{noun}. If the \texttt{predicate} is \texttt{is}, the \texttt{subject} is a \texttt{pronoun}.\\ \\
%
\textbf{Write a Prolog question to produce all correct sentences in the grammar.}\\ \\
%
You can test your program with the following examples: \\
\texttt{this is a sleeps // False} \\
\texttt{this is a likes // False} \\
\texttt{the boy likes // False} \\
\texttt{that boy likes // False} \\
\texttt{a boy the the girl // False} \\
\texttt{this is a boy // True} \\
\texttt{that is the girl // True} \\
\texttt{the girl likes a girl // True}

\newpage
\solution{\fontsize{9pt}{11pt}\verbatiminput{solutions/ex1.tex}}

\end{document}
