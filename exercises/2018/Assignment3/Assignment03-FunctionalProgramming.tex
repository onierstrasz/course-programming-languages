\documentclass [11pt, a4wide, twoside]{article}

\usepackage{times}
\usepackage{epsfig}
\usepackage{ifthen}
\usepackage{xspace}
\usepackage{fancyhdr}
\usepackage{hyperref}
\usepackage{pdfpages}
\usepackage{amsmath}
\hypersetup{
    % true means draw the links themselves colored and do not draw a bounding
    % box
    colorlinks=true,
    linkcolor= blue,
    citecolor= blue,
    filecolor=blue,
    urlcolor= blue
}

% solution switch
\newboolean{showsolution}
\setboolean{showsolution}{true} %set it either to true or false


%layout
\topmargin      -5.0mm
\oddsidemargin  6.0mm
\evensidemargin -6.0mm
\textheight 215.5mm
\textwidth      160.0mm
\parindent        1.0em
\headsep          10.3mm
\headheight        12pt
\lineskip    1pt
\normallineskip     1pt

%header
\lhead{Programming Languages \\ 2021}

\rhead{Prof. O. Nierstrasz\\
Mohammadreza Hazhirpasand, Joel Niklaus}
\lfoot{page \thepage}
\rfoot{\today}
\cfoot{}

\renewcommand{\headrulewidth}{0.1pt}
\renewcommand{\footrulewidth}{0.1pt}

\renewcommand{\thesubsection}{\arabic{subsection}}

%enumeration
\newenvironment{myitemize}{%
     \begin{itemize}
     \setlength{\itemsep}{0cm}}
     {\end{itemize}}

\newenvironment{myenumerate}{%
     \begin{enumerate} \setlength{\itemsep}{0cm}}
     {\end{enumerate}}


%solution
\ifthenelse{\boolean{showsolution}}
   {  \newcommand{\solution}[1]{
   	\noindent\underline{\textbf{Answer:}}\\[2mm]
   	 \textsl{#1}
	 \vspace{10pt}
	 \normalsize
	}
  }
  {  \newcommand{\solution}[1]{} }

\newcounter{exnum}
\def\xexercise{\fontsize{12}{10}\fontseries{bx}\selectfont}
\def\xnormal{\fontseries{m}\fontshape{n}\selectfont}


\newcommand{\exercise}[1]{%
     {\addtocounter{exnum}{1}\vskip 0.8cm{\xexercise \noindent Exercise
\arabic{exnum} (#1)} \xnormal} \vskip 0.3cm} 
 \newcommand{\aufgabe}[1]{
     {\addtocounter{exnum}{1}\vskip 0.8cm{\xexercise \noindent Aufgabe
\arabic{exnum} (#1)} \xnormal} \vskip 0.3cm} 

\pagestyle{fancy}


% ===============ABBREVIATIONS==============================
\newcommand{\eg}{\emph{e.g.,}\xspace}
\newcommand{\ie}{\emph{i.e.,}\xspace}
\newcommand{\etc}{\emph{etc.}\xspace}


\begin{document}

% title
\section*{\ifthenelse{\boolean{showsolution}}{Solution}{}\space{} Functional Programming}

% - - - - - - - - - - - - - - - - - - - - - - - - - - - - - - - - - - - - - - -

\subsection*{Instructions:}

\textbf{Solutions of the exercises are to be delivered before Thursday, the 15th of March at 10:15AM.}\\
Solutions should be placed in a separate folder with the name ``\textbf{Assignment03}''.\\
Please submit answers to all the exercises in \textbf{one} .hs file named ``assignment03.hs''.\\
Please use the provided \href{http://scg.unibe.ch/download/lectures/pl2018-exercises/assignment03.hs}{template} in which all the solutions should be written.

\subsection*{Exercise 1 (1.5 points)}

Define a function \texttt{firstNCatalan n} in Haskell that will calculate and return as the result the list which contains the first \texttt{n} \href{https://en.wikipedia.org/wiki/Catalan_number}{Catalan numbers}. Catalan numbers are calculated based on the formula \mbox{$C_{n} = \dfrac{(2n)!}{(n+1)!n!}$,} $n \ge 0$.

\vspace{0.2cm}
\solution{
\texttt{fac n\\
   \indent\indent | n == 0 = 1\\
   \indent\indent | otherwise = n * fac (n-1)\\
catalan n\\
   \indent\indent | n >= 0 = fac (2*n) / (fac n * fac (n+1))\\
firstNCatalan n = [catalan x |x <-[0..n]]
}
}

\subsection*{Exercise 2 (1.5 points)}

Define a function \texttt{perfectNumbers n m } in Haskell that returns as the result the list of all \href{https://en.wikipedia.org/wiki/Perfect_number}{perfect numbers} greater than \texttt{n} and smaller than \texttt{m}. A positive integer is \textbf{perfect} if it is equal to the sum of its proper positive factors.

\vspace{0.2cm}
\solution{
\texttt{
factors n = [x | x <- [1..n-1], mod n x == 0 ]\\
isPerfect n = sum (factors n) == n\\
perfectNumbers n m  = [x | x <- [n+1..m-1], isPerfect x]
}}

\subsection*{Exercise 3 (1.5 points)}

Define a function \texttt{insert i n l} in Haskell that returns as the result the list that contains as the first \texttt{i} elements the same ones as in the list \texttt{l}, preserving the order, followed by the element \texttt{n} on the \mbox{\texttt{i}-\textit{th}} position, and the remaining elements of the list \texttt{l}, preserving the order. In case that \texttt{i} exceeds the size of the list, the resulting list should have all the elements of the list \texttt{l}, preserving the order, and the element \texttt{n} as the last one. The index counting starts from zero.

\newpage

\vspace{0.2cm}
\solution{
\texttt{insert \_ n [] = [n]\\
insert 0 n l = n:l \\
insert i n (x:xs) = x : insert (i-1) n xs}
}

\subsection*{Exercise 4 (1.5 points)}

Define a function \texttt{indexes n l} in Haskell that returns as the result the list containing all the indexes in the list \texttt{l} where the element \texttt{n} appears. In case that \texttt{n} is not contained in the list, the function returns an empty list. The index counting starts from zero.

\vspace{0.2cm}
\solution{
\texttt{
indexes a l = indexesFrom 0 a l\\
indexesFrom i n [] = []\\
indexesFrom i n (x:xs)\\
	\indent\indent | x == n = i : indexesFrom (i+1) n xs\\
   	\indent\indent | otherwise = indexesFrom (i+1) n xs
}
}


\end{document}
