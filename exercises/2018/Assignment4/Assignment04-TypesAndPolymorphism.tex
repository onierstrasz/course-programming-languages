\documentclass [11pt, a4wide, twoside]{article}

\usepackage{times}
\usepackage{epsfig}
\usepackage{ifthen}
\usepackage{xspace}
\usepackage{fancyhdr}
\usepackage{hyperref}
\usepackage{pdfpages}
\usepackage{amsmath}

\hypersetup{
    % true means draw the links themselves colored and do not draw a bounding
    % box
    colorlinks=true,
    linkcolor= blue,
    citecolor= blue,
    filecolor=blue,
    urlcolor= blue
}

% solution switch
\newboolean{showsolution}
\setboolean{showsolution}{true} %set it either to true or false


%layout
\topmargin      -5.0mm
\oddsidemargin  6.0mm
\evensidemargin -6.0mm
\textheight 215.5mm
\textwidth      160.0mm
\parindent        1.0em
\headsep          10.3mm
\headheight        12pt
\lineskip    1pt
\normallineskip     1pt

%header
\lhead{Programming Languages \\ 2021}

\rhead{Prof. O. Nierstrasz\\
Mohammadreza Hazhirpasand, Joel Niklaus}
\lfoot{page \thepage}
\rfoot{\today}
\cfoot{}

\renewcommand{\headrulewidth}{0.1pt}
\renewcommand{\footrulewidth}{0.1pt}

\renewcommand{\thesubsection}{\arabic{subsection}}

%enumeration
\newenvironment{myitemize}{%
     \begin{itemize}
     \setlength{\itemsep}{0cm}}
     {\end{itemize}}

\newenvironment{myenumerate}{%
     \begin{enumerate} \setlength{\itemsep}{0cm}}
     {\end{enumerate}}


%solution
\ifthenelse{\boolean{showsolution}}
   {  \newcommand{\solution}[1]{
   	\noindent\underline{\textbf{Answer:}}\\[2mm]
   	 \textsl{#1}
	 \vspace{10pt}
	 \normalsize
	}
  }
  {  \newcommand{\solution}[1]{} }

\newcounter{exnum}
\def\xexercise{\fontsize{12}{10}\fontseries{bx}\selectfont}
\def\xnormal{\fontseries{m}\fontshape{n}\selectfont}


\newcommand{\exercise}[1]{%
     {\addtocounter{exnum}{1}\vskip 0.8cm{\xexercise \noindent Exercise
\arabic{exnum} (#1)} \xnormal} \vskip 0.3cm} 
 \newcommand{\aufgabe}[1]{
     {\addtocounter{exnum}{1}\vskip 0.8cm{\xexercise \noindent Aufgabe
\arabic{exnum} (#1)} \xnormal} \vskip 0.3cm} 

\pagestyle{fancy}


% ===============ABBREVIATIONS==============================
\newcommand{\eg}{\emph{e.g.,}\xspace}
\newcommand{\ie}{\emph{i.e.,}\xspace}
\newcommand{\etc}{\emph{etc.}\xspace}


\begin{document}

% title
\section*{\ifthenelse{\boolean{showsolution}}{Solution }{}Types and Polymorphism}

% - - - - - - - - - - - - - - - - - - - - - - - - - - - - - - - - - - - - - - -

\subsection*{Instructions:}

\textbf{Solutions of the exercises are to be delivered before Thursday, the 22th of March at 10:15AM.}\\
Solutions should be placed in a separate folder with the name ``\textbf{Assignment04}''.\\
Please submit answers to all the exercises in \textbf{one} text file.\\

\subsection*{Exercise 1 (3 points)}

Infer types of the functions \texttt{factors}, \texttt{isPerfect} and \texttt{insert} and say whether they are monomorphic or polymorphic functions. Justify your answer.

\begin{itemize}
\item \texttt{mod :: Int -> Int -> Int\\
factors n = [x | x <- [1..n-1], mod n x == 0 ] \\
isPerfect n = sum (factors n) == n}

\item \texttt{insert \_ n [] = [n]\\
insert 0 n l = n:l \\
insert i n (x:xs) = x : insert (i-1) n xs}
\end{itemize}

\vspace{0.2cm}
\solution{\texttt{factors :: Int -> [Int]}\\
since both \texttt{n} and \texttt{x} are arguments of the function \texttt{mod} which accepts only the \texttt{Int} arguments\\
\\
\texttt{isPerfect :: Int -> Bool}\\
since \texttt{n} is an argument of the function \texttt{factors} which accepts only the \texttt{Int} arguments,\\
and \texttt{== :: Eq a => a -> a -> Bool}\\
\\
Both functions are monomorphic.\\
\texttt{-----------------------------------------------------------------------}\\
\texttt{insert :: Int -> a -> [a] -> [a]}\\
since\\
\texttt{insert \_ n l = [n] => insert :: a->b->c->[b]}\\
\texttt{insert 0 n l = n:l => insert :: Int->b->[b]->[b]}\\
The \texttt{insert} function is polymorphic.

\texttt{mH (a, b, c) = c} \\

\texttt{mH :: (x, y, z) -> z}\\
mH is polymorphic since the three elements a, b, and c may be of any type. \\}

\subsection*{Exercise 2 (3 points)}

Infer the type of the following function and explain each of the steps.\\
\texttt{f1 f x}

	~~~~~ \texttt{| f x < 0 = []}
	
    	~~~~~ \texttt{| otherwise = x : (f1 f (f x))}


\newpage
\solution{\texttt{data Shapes = Circle Float | Rectangle Float Float} \\
\texttt{calit :: Shapes -> Float} \\
\texttt{calit (Circle r) = 2 * pi * r} \\
\texttt{calit (Rectangle a b) = a * b} \\}

\subsection*{Optional Haskell exercise (2 points)}

Write a function \texttt{deleteRepetitions l} which deletes all consecutive repetitions of elements in the list \texttt{l}.
For example, \texttt{deleteRepetitions [4, 5, 5, 2, 11, 11, 11, 2, 2]} would return as the result \texttt{[4, 5, 2, 11, 2]}.
\textbf{No built-in function for working with lists may be used. Only pattern matching is allowed.}

\vspace{0.2cm}
\solution{
\texttt{
deleteRepetitions [] = []\\
deleteRepetitions (head:[]) = [head]\\
deleteRepetitions (first:second:tail) = \\
	   \indent\indent if first == second\\
           \indent\indent then deleteRepetitions (second:tail)\\
           \indent\indent else first : deleteRepetitions (second:tail)\\
}
}


\end{document}
