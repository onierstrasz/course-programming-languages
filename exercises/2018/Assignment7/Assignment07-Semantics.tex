\documentclass [11pt, a4wide, twoside]{article}

\usepackage{times}
\usepackage{epsfig}
\usepackage{ifthen}
\usepackage{xspace}
\usepackage{fancyhdr}
\usepackage{moreverb}
\usepackage{amsmath}
\usepackage{amsthm}
\usepackage[usenames, dvipsnames]{color}
\usepackage{stmaryrd}

% solution switch
\newboolean{showsolution}
\setboolean{showsolution}{true}


%layout
\topmargin      -5.0mm
\oddsidemargin  6.0mm
\evensidemargin -6.0mm
\textheight 215.5mm
\textwidth      160.0mm
\parindent        1.0em
\headsep          10.3mm
\headheight        12pt
\lineskip    1pt
\normallineskip     1pt

%header
\lhead{Programming Languages \\ 2021}

\rhead{Prof. O. Nierstrasz\\
Mohammadreza Hazhirpasand, Joel Niklaus}
\lfoot{page \thepage}
\rfoot{\today}
\cfoot{}

\renewcommand{\headrulewidth}{0.1pt}
\renewcommand{\footrulewidth}{0.1pt}

\renewcommand{\thesubsection}{\arabic{subsection}}

%enumeration
\newenvironment{myitemize}{%
     \begin{itemize}
     \setlength{\itemsep}{0cm}}
     {\end{itemize}}

\newenvironment{myenumerate}{%
     \begin{enumerate} \setlength{\itemsep}{0cm}}
     {\end{enumerate}}


%solution
\ifthenelse{\boolean{showsolution}}
   {  \newcommand{\solution}[1]{
   	\noindent\underline{\textbf{Answer:}}\\[2mm]
   	 \textsl{#1}
	 \vspace{10pt}
	 \normalsize
	}
  }
  {  \newcommand{\solution}[1]{} }

\newcounter{exnum}
\def\xexercise{\fontsize{12}{10}\fontseries{bx}\selectfont}
\def\xnormal{\fontseries{m}\fontshape{n}\selectfont}


\newcommand{\exercise}[1]{%
     {\addtocounter{exnum}{1}\vskip 0.8cm{\xexercise \noindent Exercise
\arabic{exnum} (#1)} \xnormal} \vskip 0.3cm} 
 \newcommand{\aufgabe}[1]{
     {\addtocounter{exnum}{1}\vskip 0.8cm{\xexercise \noindent Aufgabe
\arabic{exnum} (#1)} \xnormal} \vskip 0.3cm} 

\pagestyle{fancy}


% ===============ABBREVIATIONS==============================
\newcommand{\eg}{\emph{e.g.,}\xspace}
\newcommand{\ie}{\emph{i.e.,}\xspace}
\newcommand{\etc}{\emph{etc.}\xspace}


\begin{document}

% title
\section*{\ifthenelse{\boolean{showsolution}}{Solution }{}\xspace{}Programming Language Semantics}

\subsection*{Instructions:}

\textbf{Solutions of the exercises are to be delivered before Thursday, the 26th of April at 10:15AM.}\\
Solutions should be placed in a separate folder with the name ``\textbf{Assignment07}''.\\
Please submit answers to all the exercises in \textbf{one} text file.\\

% - - - - - - - - - - - - - - - - - - - - - - - - - - - - - - - - - - - - - - - - - - - - - - - - - - - - - - - - - - - - - - - - - - -
\subsection*{Exercise 1 (3 points)}

Extend the abstract syntax and the semantic functions \textbf{P}, \textbf{S} and \textbf{E} of the Calculator Language, defined at the lecture hours, in order to include the possibility of subtraction and division of two expressions. In case the divider is zero, the result should be the string ``NOT A NUMBER''.

\vspace{0.5cm}
\solution{\texttt{factors :: Int -> [Int]}\\
since both \texttt{n} and \texttt{x} are arguments of the function \texttt{mod} which accepts only the \texttt{Int} arguments\\
\\
\texttt{isPerfect :: Int -> Bool}\\
since \texttt{n} is an argument of the function \texttt{factors} which accepts only the \texttt{Int} arguments,\\
and \texttt{== :: Eq a => a -> a -> Bool}\\
\\
Both functions are monomorphic.\\
\texttt{-----------------------------------------------------------------------}\\
\texttt{insert :: Int -> a -> [a] -> [a]}\\
since\\
\texttt{insert \_ n l = [n] => insert :: a->b->c->[b]}\\
\texttt{insert 0 n l = n:l => insert :: Int->b->[b]->[b]}\\
The \texttt{insert} function is polymorphic.

\texttt{mH (a, b, c) = c} \\

\texttt{mH :: (x, y, z) -> z}\\
mH is polymorphic since the three elements a, b, and c may be of any type. \\}


% - - - - - - - - - - - - - - - - - - - - - - - - - - - - - - - - - - - - - - - - - - - - - - - - - - - - - - - - - - - - - - - - - - -
\subsection*{Exercise 2 (3 points)}

Consider a language of binary numbers. The number \textbf{'111'} is intended to denote the natural number \textbf{7}. Define the syntax, the semantic functions and the domain of this language. As a test evaluate \textbf{'10101'}.

\vspace{0.5cm}
\solution{and $\equiv$ $\lambda$~x y .~x y x \\
	\indent\indent // if the first argument is \texttt{False}, the function should return \texttt{False}, \ie the first argument; \\
	\indent\indent // if the first argument is \texttt{True}, the function should return the second argument\\
or $\equiv$ $\lambda$~x y .~x x y \\
	\indent\indent // if the first argument is \texttt{True}, the function should return \texttt{True}, \ie the first argument; \\
	\indent\indent // if the first argument is \texttt{False}, the function should return the second argument\\

\texttt{True and False = False}\\
\indent\indent ($\lambda$~x y.~x y x)($\lambda$~x y.~x)($\lambda$~x y.~y) = \\
\indent\indent ($\lambda$~x y.~x)($\lambda$~x y.~y)($\lambda$~x y.~x) = \\
\indent\indent ($\lambda$~x y.~y) $\equiv$ \texttt{False} \\

\texttt{True or False = True}\\
\indent\indent ($\lambda$~x y.~x x y)($\lambda$~x y.~x)($\lambda$~x y.~y) = \\
\indent\indent ($\lambda$~x y.~x)($\lambda$~x y.~x)($\lambda$~x y.~y) = \\
\indent\indent ($\lambda$~x y.~x) $\equiv$ \texttt{True} \\}

\end{document}
