\textbf{NB: For this exercise, you are allowed to use only arithmetical built-in Haskell functions!}
\begin{itemize}
%
\item (4 points) Write a Haskell function \texttt{magic n} which returns the magic number of the argument \texttt{n}. The magic number of a natural number \texttt{n} is equal to the product of its digits.
\vspace{9cm}
%
\item (4 points) Write a Haskell function \texttt{number s} which returns the number represented by the elements of the list \texttt{s}. Consider that elements of the list are only natural numbers smaller than 10. For example, \texttt{number [1,2,3]} will return as the result the number \texttt{123}. 
%
\newpage
\item (5 points) Infer the type of the following function and explain your steps. Is the function monomorphic or polymorphic? Explain.
\begin{verbatim}
apply f g [] = []
apply f g (x:xs)
   | f x < 0 = (g x) : apply f g xs
   | otherwise = (f x) : apply f g xs
\end{verbatim}
\end{itemize}