\noindent
Create a finite collection of definite clause grammar rules to check whether a sentence is grammatically
correct. A sentence can be composed of the following words:
%
A sentence can be composed of the following words:
\begin{description}
	\item [\texttt{article}] \texttt{a}, \texttt{the}
	\item [\texttt{noun}] \texttt{girl}, \texttt{girls}.
	\item [\texttt{verb}] \texttt{play}, \texttt{plays}.
\end{description}
%
A sentence must be in the form \texttt{subject-predicate}.
%
\begin{itemize}
\item \texttt{subject} is formed out of an \texttt{article} and a \texttt{noun}. For example, \texttt{a girl}.
\item \texttt{predicate} is a \texttt{verb}
\end{itemize}
%
The sentence should be grammatically correct in a sense that the article \texttt{a} cannot be used in front of a noun in plural. If a \texttt{subject} is in plural, the following \texttt{verb} must be \texttt{play}, and if the \texttt{subject} is in singular, the following verb should be \texttt{plays}.
\\ \\ 
%
\textbf{Write a Prolog question to produce all correct sentences in the grammar.}\\ \\
%
You can test your program with the following examples: \\
\texttt{a girl plays // True} \\
\texttt{a girl play // False} \\
\texttt{the girls plays // False} \\
\texttt{the girls play // True} \\
\texttt{girls plays // False} \\
\texttt{girls play // False} \\
\texttt{a girls play // False} \\
