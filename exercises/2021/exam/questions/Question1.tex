% 2014 Exam
Answer the following questions. Do not write more than 3 sentences each. Each question is worth 2 points.
\begin{enumerate}

% General
\item Distinguish the key characteristics of the functional and logic programming styles.
\vspace{5cm}

% PostScript
\item How is \texttt{(1 + 3) * 2} computed in a stack-based language like PostScript? Note the contents of the stack after each operation.
\vspace{5cm}

% Functional Languages
\item Why don't pure functional languages provide loop constructs?
\vspace{5cm}

\newpage
% Monomorphism vs polymorphism
\item
What is the difference between monomorphic and polymorphic types? Is the following Haskell function monomorphic or polymorphic? Why?\\
\texttt{map f [ ] = [ ]}\\
\texttt{map f (x:xs) = f x : map f xs} 
\vspace{5cm}

%\newpage

% Fixed points and Recursion
% Removed because too hard
%\item Explain how it is possible to use the concept of recursion in the $\lambda$-calculus.
%\vspace{6cm}

% Lazy evaluation
\item Why is normal order evaluation called lazy evaluation? \\
Consider the expression: \texttt{sqr n = n * n} and compute the normal order evaluation of \texttt{sqr (7-3)}
\vspace{5cm}

% Syntax and semantics
\item What is the difference between syntax and semantics?
\vspace{5cm}

\newpage

% Object, types and prototypes
\item List differences between a class-based and a prototype-based programming language.
\vspace{5cm}

% JavaScript
\item What happens in JavaScript if we assign a variable without using the command ``var''? What problem can it cause? 
\vspace{5cm}

% Prolog
\item What does the closed world assumption in Prolog state?

%\item Is there a difference between logical negation \texttt{$\neg$} and %\texttt{not} operator implemented using cut and fail? Explain.  \\
%\texttt{not(X) :- X!, fail. } \\
%\texttt{not(\_).}
%\vspace{2.5cm}

\end{enumerate}