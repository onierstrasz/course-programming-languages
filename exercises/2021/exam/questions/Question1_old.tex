Answer the following questions. Do not write more than 3 sentences. Each question is worth 2 points.
\begin{enumerate}
%general
\item Name three programming paradigms/styles and note their unique characteristics.
\vspace{5cm}

%PostScript
%\item How is \texttt{(1 + 3) * 2} computed in a stack-based language like PostScript? Note the contents of the stack after each operation.
%\vspace{4cm}

% side effect
%\item Can there be a side effect in a pure functional programming language? Justify your answer.
%\vspace{4cm}

%curried functions
\item Are two following Haskell \texttt{times} functions equivalent? Justify your answer.\\
\texttt{times x y = x * y}\\
\texttt{times (x,y) = x * y}
\vspace{5cm}

%monomorphism vs polymorphism
\item
What is the difference between monomorphic and polymorphic type? Is the following Haskell function monomorphic or polymorphic? Why?\\
\texttt{sum [] = 0}\\
\texttt{sum (x:xs) = x + sum xs} 
\vspace{5cm}

\newpage
% lambda calculus
\item What is a normal form of a $\lambda$ expression? How does one reach it? Does the expression\\
\texttt{($\lambda$ x. y)($\lambda$ f. ($\lambda$ x. f (x x))($\lambda$ x. f (x x)))}\\
have a normal form? Justify your answer.
\vspace{6cm}

%fixed point and recursion
\item Is it possible to define a recursive expression in $\lambda$ calculus? Justify your answer.
\vspace{6cm}

%syntax and semantics
\item What is the difference between abstract and concrete syntax?
\vspace{6cm}

\newpage
\item What is static semantics and what is dynamic semantics?
\vspace{6cm}

% Object, types and prototypes
%\item List differences between a class-based and a prototype-based programming language.
%\vspace{4cm}

\item Explain what are parametric polymorphism and coercion and provide an example for each of them.
\vspace{6cm}

%Prolog
\item How are questions answered in a Prolog program? For example, \texttt{mother(charles, M)}.
\end{enumerate}