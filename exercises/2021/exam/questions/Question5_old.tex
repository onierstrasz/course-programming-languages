We represent non-negative integers with the following Lambda expressions:
%
\begin{align*}
	0 & \equiv \lambda f~.~\lambda x~.~x \\
	1 & \equiv \lambda f~.~\lambda x~.~f x \\
	2 & \equiv \lambda f~.~\lambda x~.~f (f x) \\
	& \vdots  \\
	n & \equiv \lambda f~.~\lambda x~.~f^{n}x
\end{align*}
%
Suppose you have defined the function \textbf{if} and the operations \textbf{times}, \textbf{pred} and \textbf{isOne}. Consider the following recursive (and hence not valid) definition for the factorial calculation:
%
\begin{equation*}
\textbf{fact} = \lambda n.~\textbf{if} ~(\textbf{isOne}~ n) ~\textbf{1}~ (\textbf{times}~ n~(\textbf{fact}~(\textbf{pred}~n)))
\end{equation*}
%
To do:
\begin{enumerate}
\item (4 points) Translate the \textbf{fact} definition into a proper definition, i.e., using the Y combinator.
\vspace{6cm}
\item (8 points) Write down the reduction sequence to demonstrate that factorial of 3 is 6.
\end{enumerate}
%
