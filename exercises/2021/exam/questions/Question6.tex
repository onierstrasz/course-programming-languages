\noindent
Create a finite collection of definite clause grammar rules to check whether a sentence is grammatically
correct. A sentence can be composed of the following words:
%
\begin{description}
	\item [\texttt{Subject pronoun:}] \texttt{he}, \texttt{she}, \texttt{you}.
	\item [\texttt{Wh pronoun:}] \texttt{what}, \texttt{where}.
	\item [\texttt{Verb:}] \texttt{play}, \texttt{paint}.
	\item [\texttt{Auxiliary:}] \texttt{did}.
    \item [\texttt{Verb, past tense:}] \texttt{played}, \texttt{painted}.

\end{description}
%
A sentence must be in the form \texttt{subject-predicate} or \texttt{wh pronoun-auxiliary-subject pronoun-verb?}.
%
\\
Predicate means either an auxiliary or verb past tense.
\\
%
Note: The resulting sentence must be a simple past tense, either in the form of a sentence or question.
\\ \\
You can test your program with the following examples: \\
\texttt{she painted // True} \\
\texttt{you painted // True} \\
\texttt{he painted // True} \\
\texttt{she played // True} \\
\texttt{she did // True } \\
\texttt{what did she paint // True} \\
\texttt{where did he play // True} \\
\texttt{he did painted // False} \\
\texttt{he what painted // False} \\
\texttt{what did she painted // False} \\
\texttt{where did you played // False} \\

\newpage
\begin{enumerate}
    \item (10 points) Write all necessary definite clause grammar rules.
    
    \vspace{14cm}
    
    \item (3 points) Write a Prolog question to produce all the correct sentences in the grammar.
\end{enumerate}




