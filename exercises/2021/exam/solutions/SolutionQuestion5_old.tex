If we abstract the name \textbf{fact}, we get : \textbf{t} = $\lambda$ f . $\lambda$ n .\textbf{if} (\textbf{isOne} n) \textbf{1} (\textbf{times} n (f (\textbf{pred} n))).\\
%
(Y t)  is a non-recursive equivalent of the above \textbf{fact} definition.\\
%
\noindent
( Y t ) 3 $\equiv$ /* Fixpoint Theorem tells us that Y t = t ( Y t ) */\\
\noindent
t ( Y t ) 3 $\equiv$\\
\noindent
($\lambda$ f~.~$\lambda$ n~.~\textbf{if}~ (\textbf{isOne}~ n) ~\textbf{1}~ (\textbf{times}~ n~ (f~ (\textbf{pred}~ n))))( Y t ) 3 $\equiv$ /* f =  Y t. n = 3. */\\
\noindent
\textbf{if}~ (\textbf{isOne}~ 3) ~\textbf{1}~ (\textbf{times}~ 3~ ((Y t)~ (\textbf{pred}~ 3))) $\equiv$ /* \textbf{isOne}~ 3 = False */\\
\noindent
\textbf{times}~ 3~ ((Y t)~2) $\equiv$\\
\noindent
\textbf{times}~ 3~ (t (Y t)~2) $\equiv$\\
\noindent
\textbf{times}~ 3~(($\lambda$ f~.~$\lambda$ n~.~\textbf{if}~ (\textbf{isOne}~ n) ~\textbf{1}~ (\textbf{times}~ n~ (f~ (\textbf{pred}~ n))))( Y t ) 2) $\equiv$\\
\noindent
\textbf{times}~ 3~(\textbf{if}~ (\textbf{isOne}~ 2) ~\textbf{1}~ (\textbf{times}~ 2~ ((Y t)~ (\textbf{pred}~ 2)))) $\equiv$ \\
\noindent
\textbf{times}~ 3~(\textbf{times}~ 2~ ((Y t)~1)) $\equiv$ \\
\noindent
\textbf{times}~ 3~(\textbf{times}~ 2~ (t (Y t)~1)) $\equiv$ \\
\noindent
\textbf{times}~ 3~(\textbf{times}~ 2~ (($\lambda$ f~.~$\lambda$ n~.~\textbf{if}~ (\textbf{isOne}~ n) ~\textbf{1}~ (\textbf{times}~ n~ (f~ (\textbf{pred}~ n))))(Y t)~1)) $\equiv$ \\
\noindent
\textbf{times}~ 3~(\textbf{times}~ 2~ (~\textbf{if}~ (\textbf{isOne}~ 1) ~\textbf{1}~ (\textbf{times}~ 1~ ((Y t)~ (\textbf{pred}~ 1))))) $\equiv$ \\
\noindent
\textbf{times}~ 3~(\textbf{times}~ 2~1) $\equiv$ \\
\noindent
\textbf{times}~ 3~2 $\equiv$ \\
\noindent
6