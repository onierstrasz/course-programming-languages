\documentclass [11pt, a4wide, twoside]{article}


\usepackage{hyperref}
\usepackage{times}
\usepackage{epsfig}
\usepackage{ifthen}
\usepackage{xspace}
\usepackage{fancyhdr}
\usepackage{moreverb}
\usepackage{amsmath}
\usepackage{hyperref} % for figures
\usepackage{listings} % for Java code
\usepackage{color} % for Java code

%========================JAVA CODE========================
\definecolor{source}{gray}{0.9}
\lstset{
	% characters
	tabsize=3,
	upquote=true, % straight quote; requires textcomp package
	escapechar={!},
	keepspaces=true,
	breaklines=false,
	alsoletter={:},
	breakautoindent=true,
	columns=fullflexible,
	showspaces=false,
	showstringspaces=false,
	basicstyle=\small\ttfamily,
	% background
	frame=single,
   	framerule=0pt,
	backgroundcolor=\color{source},
	% numbering
	numbersep=10pt,
	stepnumber=1,
	numberstyle=\tiny,
	numberfirstline=true,
	% captioning
	captionpos=b,
	numberbychapter=false}
\definecolor{javared}{rgb}{0.6,0,0} % for strings
\definecolor{javagreen}{rgb}{0.25,0.5,0.35} % comments
\definecolor{javapurple}{rgb}{0.5,0,0.35} % keywords
\definecolor{javadocblue}{rgb}{0.25,0.35,0.75} % javadoc
%Java listing
\lstdefinestyle{Java}{
	language=Java,
	keywordstyle=\color{javapurple}\bfseries,
	stringstyle=\color{javared},
	commentstyle=\color{javagreen},
	morecomment=[s][\color{javadocblue}]{/**}{*/}
}
\newcommand{\jlct}{\lstinline[backgroundcolor=\color{white},style=Java]}
%========================================================
\hypersetup{
    % true means draw the links themselves colored and do not draw a bounding
    % box
    colorlinks=true,
    linkcolor= blue,
    citecolor= blue,
    filecolor=blue,
    urlcolor= blue
}

% solution switch
\newboolean{showsolution}
\setboolean{showsolution}{true} %set it either to true or false


%layout
\topmargin      -5.0mm
\oddsidemargin  6.0mm
\evensidemargin -6.0mm
\textheight 215.5mm
\textwidth      160.0mm
\parindent        1.0em
\headsep          10.3mm
\headheight        12pt
\lineskip    1pt
\normallineskip     1pt

%header
\lhead{Programming Languages \\ 2021}

\rhead{Prof. O. Nierstrasz\\
Mohammadreza Hazhirpasand, Joel Niklaus}
\lfoot{page \thepage}
\rfoot{\today}
\cfoot{}

\renewcommand{\headrulewidth}{0.1pt}
\renewcommand{\footrulewidth}{0.1pt}

\renewcommand{\thesubsection}{\arabic{subsection}}

%enumeration
\newenvironment{myitemize}{%
     \begin{itemize}
     \setlength{\itemsep}{0cm}}
     {\end{itemize}}

\newenvironment{myenumerate}{%
     \begin{enumerate} \setlength{\itemsep}{0cm}}
     {\end{enumerate}}


%solution
\ifthenelse{\boolean{showsolution}}
   {  \newcommand{\solution}[1]{
   	\noindent\underline{\textbf{Answer:}}\\[2mm]
   	 \textsl{#1}
	 \vspace{10pt}
	 \normalsize
	}
  }
  {  \newcommand{\solution}[1]{} }

\newcounter{exnum}
\def\xexercise{\fontsize{12}{10}\fontseries{bx}\selectfont}
\def\xnormal{\fontseries{m}\fontshape{n}\selectfont}


\newcommand{\exercise}[1]{%
     {\addtocounter{exnum}{1}\vskip 0.8cm{\xexercise \noindent Exercise
\arabic{exnum} (#1)} \xnormal} \vskip 0.3cm} 
 \newcommand{\aufgabe}[1]{
     {\addtocounter{exnum}{1}\vskip 0.8cm{\xexercise \noindent Aufgabe
\arabic{exnum} (#1)} \xnormal} \vskip 0.3cm} 

\pagestyle{fancy}


% ===============ABBREVIATIONS==============================
\newcommand{\eg}{\emph{e.g.,}\xspace}
\newcommand{\ie}{\emph{i.e.,}\xspace}
\newcommand{\etc}{\emph{etc.}\xspace}


\begin{document}

% title
\section*{\ifthenelse{\boolean{showsolution}}{Solution}{}\space{} Logic programming}

% - - - - - - - - - - - - - - - - - - - - - - - - - - - - - - - - - - - - - - -

\begin{myitemize}
\item Exercises are given every week on the PL page of the SCG website \\ (\url{http://scg.unibe.ch/teaching/pl})
\item Solutions to each assignment must be sent to \textbf{mohammadreza.hazhirpasand@inf.unibe.ch}
\item The solutions of the assignments are to be delivered before every Thursday at 11 PM. Solutions handed in later than the specified time will not be accepted. In case of serious reasons send an e-mail to  \textbf{mohammadreza.hazhirpasand@inf.unibe.ch}
\end{myitemize}


\subsection*{Exercise (6 points)}


\begin{enumerate}
\item We will build a genealogy that covers relations in a family. Consider a genealogy database consisting of the following predicates: (3 pts)
%
\begin{verbatim}
  female(X), male(X), parent(X,Y)
\end{verbatim}
%
\noindent Define rules allowing you to determine the following relations:
%
\begin{verbatim}
  grandfather(X,Y), grandmother(X,Y), grandparent(X,Y),
  grandson(X,Y), granddaughter(X,Y), grandchild(X,Y)
\end{verbatim}

\vspace{0.5cm}

\solution{\input{solutions/db2}}

% - - - - - - - - - - - - - - - - - - - - - - - - - - - - - - - - - - - - - - - - - - - - - - - - - - - - - - - - - - - - - - - - - - -


\item Using the following weekly schedule, write the necessary facts and a rule in order to output days as well as the associated programs and difficulty levels. (3 pts)

\begin{verbatim}
  monday - english - simple
  tuesday - programming - medium
  tuesday - ai - hard
  wednesday - hacking - hard
  thursday - networking - medium
  friday - pl - easiest 
\end{verbatim}

\vspace{0.5cm}

\solution{\input{solutions/db}}

\end{enumerate}

\end{document}
