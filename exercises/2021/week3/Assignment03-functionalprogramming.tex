\documentclass [11pt, a4wide, twoside]{article}

\usepackage{times}
\usepackage{epsfig}
\usepackage{ifthen}
\usepackage{xspace}
\usepackage{fancyhdr}
\usepackage{hyperref}
\usepackage{pdfpages}
\usepackage{amsmath}
\hypersetup{
    % true means draw the links themselves colored and do not draw a bounding
    % box
    colorlinks=true,
    linkcolor= blue,
    citecolor= blue,
    filecolor=blue,
    urlcolor= blue
}

% solution switch
\newboolean{showsolution}
\setboolean{showsolution}{true} %set it either to true or false


%layout
\topmargin      -5.0mm
\oddsidemargin  6.0mm
\evensidemargin -6.0mm
\textheight 215.5mm
\textwidth      160.0mm
\parindent        1.0em
\headsep          10.3mm
\headheight        12pt
\lineskip    1pt
\normallineskip     1pt

%header
\lhead{Programming Languages \\ 2021}

\rhead{Prof. O. Nierstrasz\\
Mohammadreza Hazhirpasand, Joel Niklaus}
\lfoot{page \thepage}
\rfoot{\today}
\cfoot{}

\renewcommand{\headrulewidth}{0.1pt}
\renewcommand{\footrulewidth}{0.1pt}

\renewcommand{\thesubsection}{\arabic{subsection}}

%enumeration
\newenvironment{myitemize}{%
     \begin{itemize}
     \setlength{\itemsep}{0cm}}
     {\end{itemize}}

\newenvironment{myenumerate}{%
     \begin{enumerate} \setlength{\itemsep}{0cm}}
     {\end{enumerate}}


%solution
\ifthenelse{\boolean{showsolution}}
   {  \newcommand{\solution}[1]{
   	\noindent\underline{\textbf{Answer:}}\\[2mm]
   	 \textsl{#1}
	 \vspace{10pt}
	 \normalsize
	}
  }
  {  \newcommand{\solution}[1]{} }

\newcounter{exnum}
\def\xexercise{\fontsize{12}{10}\fontseries{bx}\selectfont}
\def\xnormal{\fontseries{m}\fontshape{n}\selectfont}


\newcommand{\exercise}[1]{%
     {\addtocounter{exnum}{1}\vskip 0.8cm{\xexercise \noindent Exercise
\arabic{exnum} (#1)} \xnormal} \vskip 0.3cm} 
 \newcommand{\aufgabe}[1]{
     {\addtocounter{exnum}{1}\vskip 0.8cm{\xexercise \noindent Aufgabe
\arabic{exnum} (#1)} \xnormal} \vskip 0.3cm} 

\pagestyle{fancy}


% ===============ABBREVIATIONS==============================
\newcommand{\eg}{\emph{e.g.,}\xspace}
\newcommand{\ie}{\emph{i.e.,}\xspace}
\newcommand{\etc}{\emph{etc.}\xspace}


\begin{document}

% title
\section*{\ifthenelse{\boolean{showsolution}}{Solution}{}\space{} Functional Programming}

% - - - - - - - - - - - - - - - - - - - - - - - - - - - - - - - - - - - - - - -

\begin{myitemize}
\item Exercises are given every week on the PL page of the SCG website \\ (\url{http://scg.unibe.ch/teaching/pl})
\item Solutions to each assignment must be sent to \textbf{mohammadreza.hazhirpasand@inf.unibe.ch}
\item The solutions of the assignments are to be delivered before every Thursday at 5 PM. Solutions handed in later than the specified time will not be accepted. In case of serious reasons send an e-mail to  \textbf{mohammadreza.hazhirpasand@inf.unibe.ch}
\end{myitemize}


\subsection*{Exercise (6 points)}

\begin{myitemize}

\item Explain why the following piece of code does not raise an error. (1 pts)

\emph{
func1 5 z = 33 \\
func1 y z = y \\
func1 50 (sqrt(-5)) \\
-- output is 50
}
\\

\solution{
This is allowed because of lazy evaluation. The incorrect argument here is not going to be evaluated.}


\item Define the following small program in three different functions with pattern matching,  guards, and lambda expression. (1.5 pts)

\emph{
if n = 0 then \\
return -1 \\
else \\ 
return n * 2 \\
}
\\

%\textbf{Answer:}
%(s$\backslash$  -\textgreater  if s == 0 then -1 else s * 2) 0 \\
%
%gpdef n | n == 0 = -1 \\
%             | n /= 0 = n * 2 \\
%
%pmdef 0 = -1 \\
%pmdef n = n * 2 \\

\solution{
\texttt{(s$\backslash$  -\textgreater if s == 0 then -1 else s * 2) 0 \\
----\\
gpdef n | n == 0 = -1 \\
\indent\indent \indent\indent      | n /= 0 = n * 2 \\
----\\
pmdef 0 = -1 \\
pmdef n = n * 2 \\
}
}



\item Define a function that accepts a list as an argument and returns the sum of all the members of the given list. (1.5 pts)

\textbf{Answer:}
mh8 (x) = if x == [] then 0 else head x + mh8(tail(x))


\newpage


\item Define a function \texttt{firstNCatalan n} in Haskell that calculates and returns the result as a list containing the first \texttt{n} \href{https://en.wikipedia.org/wiki/Catalan_number}{Catalan numbers}. Catalan numbers are calculated based on the formula \mbox{$C_{n} = \dfrac{(2n)!}{(n+1)!n!}$,} $n \ge 0$. (2 pts)

\vspace{0.2cm}
\solution{
\texttt{fac n\\
   \indent\indent | n == 0 = 1\\
   \indent\indent | otherwise = n * fac (n-1)\\
catalan n\\
   \indent\indent | n >= 0 = fac (2*n) / (fac n * fac (n+1))\\
firstNCatalan n = [catalan x |x <-[0..n]]
}
}

\end{myitemize}
\end{document}
